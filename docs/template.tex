\documentclass[11pt, a4paper]{article}

\usepackage[left=2cm,text={17cm, 24cm},top=3cm]{geometry}
\usepackage{times}
\usepackage[czech]{babel}
\PassOptionsToPackage{hyphens}{url}\usepackage[hidelinks]{hyperref}

\begin{document}

\begin{titlepage}
    \begin{center}
        \textsc{\Huge Vysoké učení technické v Brně\\\vspace{0.5em}\huge Fakulta informačních technologií}
        
        \vspace{\stretch{0.382}}
        {\LARGE Typografie a publikování -- 4. projekt\\\vspace{0.5em}}
        {\Huge Bibliografické citácie}
        \vspace{\stretch{0.618}}
        
    \end{center}
    {\Large 16. dubna 2024 \hfill Lukáš Katona}
\end{titlepage}

\newpage

\section{Vznik a vývoj písma}

Písmo je jeden z najpodstatnejších vynálezov ľudstva. Je to jedna z vecí, ktoré nás odlišujú od ostatných tvorov, no prečo je vlastne tak podstatné?
Vďaka písmu vieme komunikovať naprieč priestorom aj časom. Uchováva informácie pre ďalšie generácie, viz \cite{Balko}. Bez písma by sme ako ľudstvo neboli tam kde sme. Písmo môžeme rozdeliť na 4 základné kategórie a to obrázkové písmo, slovné písmo, slabičné písmo a hláskové písmo, viz \cite{gelb1963study}.

\subsection{Obrázkové písmo}

Obrázkové písmo bol prvý spôsob ako ľudia zaznamenávali informácie. Pigmentami z podrvených rastlín, chrobákov a iných materiálov vytvárali jaskynné maľby, ktorých účelom bolo zdokumentovať udalosti ako napríklad lov či spoločenské akcie, viz \cite{clottes2016paleolithic}. Jedny z najstarších jaskynných malieb sa našli v Španielsku a vedci odhadujú, že sú približne 65 000 rokov staré, viz \cite{Greshko_2018}.

\subsection{Slovné písmo}

Po čase sa niektoré obrázky začali opakovať a ich význam zovšeobecnel. Obrázky sa začali usporadúvať do viet, kde jeden znak predstavoval jedno slovo. Asi najznámejším slovným písmom na svete sú Egyptské hieroglyfy. Slovo hieroglyf sa skladá z gréckych slov hieros (posvätný) a glyphē (rezba), viz \cite{Online_Etymology_Dictionary}. Medzi najviac zachované texty patria tie, ktoré zdobia hrobku KV17 v egyptskom Údolí kráľov. Ide o náboženské texty popisujúce cestu duše zosnulého cez podsvetie, viz \cite{Lull_Painted_Tomb}.

\subsection{Slabičné písmo}

Ďalším krokom v evolúcii písma boli slabiky. Ľudia si uvedomili, že slová v reči nie je to jediné čo sa opakuje, ale že aj samotné slová obsahujú časti, ktoré znejú rovnako. Samotný znak teda prestáva niesť konkrétny význam, ale je iba reprezentáciou výslovnosti danej časti slova, viz \cite{Marek2013thesis}. Príkladom takéhoto písma sú napríklad japonská hiragana a katakana. Hiragana sa používa na domáce slová a katakana na tie prevzaté z cudzích jazykov, kvôli rozdielom vo výslovnosti. Japonci používajú dokonca tri typy písma, tým posledným je kanji, ktoré prebrali z Číny. Toto písmo však reprezentuje celé slová a tak nepatrí medzi slabičné, ale slovné písmo, viz \cite{Harun_Biduri_2024}.

\subsection{Hláskové písmo}

Istú formu slabičného písma používali aj Feničania. Dalo by sa povedať, že fenické písmo bolo predchodcom hláskového písma. Aj keď ešte nemali samostatné symboly pre všetky hlásky, teda zvlášť pre spoluhlásky aj smohlásky, využívali značenie, ktoré menilo výslovnosť danej slabiky, viz \cite{gelb1963study}. Dôležitosť samohlások ako samostatných symbolov si uvedomili až Gréci, ktorí od Feničanov prebrali až 16 znakov úplne bez zmeny výslovnosti, a zvyšné znaky fenického písma využili na samohlásky. Taktiež vytvorili tri nové spoluhláskové tvary ph, kh a ps. Grécka abeceda sa ustálila na 24 znakov až okolo siedmeho a šiesteho storočia pred naším letopočtom keď z nej odobrali písmená digamma a koppa, viz \cite{SEMERADOVA2014thesis}.

Na rozdiel od jeho predchodcov má takéto modulárne písmo výhodu popísať nie len veci konkrétne, ale aj abstrakné, myšlienky, pocity, úvahy. Do veľmi pomohlo rozvoju vied ako je napríklad matematika, fyzika, astrológia a ďalšie. Alexander Veľký mal sen vytvoriť jedno miesto kde by sa zhromaždila všetka múdrosť sveta. A tak v Alexandrii, novom hlavnom meste Egypta, okolo roku 330 pred naším letopočtom založil Alexandrijsku knižnicu. Nevie sa to presne, no odhaduje sa, že táto impozantná stavba pod svojoch strechou v~čase jej najväčšieho úspechu zhromažďovala až okolo 70\,000 kníh, viz \cite{Alexandria_Library}. Väčšina z nich bohužial zhorela pri požiari čo spôsobilo pre vedcov dnešnej doby nenahraditeľnú strátu informácií o našej minulosti. Lebo aj keď by mohlo písmo uchovať informácie naveky, vždy bude limitované nedokonalosťou prostriedkov akými je zaznamenané.

\bibliographystyle{czechiso}
\bibliography{reference}
    
\end{document}